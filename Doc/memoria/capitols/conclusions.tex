\chapter{Conclusions}

%TODO: Parlar de com els resultats de l'algorisme són una fundació per fer una detecció automàtica de tot tipus d'incoherències

L'objectiu d'aquest projecte ha estat desenvolupar un algorisme per comparar descripcions textuals informals i models BPMN formals. Podem concloure que l'objectiu principal s'ha assolit satisfactòriament. L'algorisme desenvolupat és capaç d'agrupar parelles heterogènies de models i textos (secció \ref{sec:experiments-aparellament}) i es pot utilitzar com una eina per assistir a la detecció d'inconsistències entre els dos tipus de documentació. 

Tornant als objectius específics del treball (secció \ref{sec:introduccio-el_problema}), podem dir que s'han assolit amb un alt nivell de satisfacció. Alguns dels objectius addicionals de l'apartat \ref{sec:introduccio-abast} es plantegen com a treball futur, mentre que la resta s'han dut a terme dins el període establert.

L'utilització de tècniques de NLP és un dels aspectes principals del projecte. Tot el text, tant del model com del document, s'ha analitzat amb Freeling. S'ha fet ús extensiu dels textos analitzats a la fase d'extracció de característiques. D'altra banda, el tractament de models BPMN també ha estat un punt molt important en la implementació final, incorporant tècniques de tractament de grafs i l'algorisme dels \emph{Behavioral Profiles} (secció \ref{sec:enfoc-ordre-model}). Es pot dir que aquest projecte integra els camps \emph{Business Process Management} i el \emph{Natural Language Processing} en un mateix sistema de manera satisfactòria.

Tal com s'havia plantejat, l'enfoc de l'algorisme es basa en crear una representació homogènia amb vectors de característiques que redueixin a la mateixa representació el model BPMN i la descripció textual. Aquest enfoc ha permès facilitar substancialment la tasca de trobar una puntuació de similaritat. 

Respecte establir un ordre entre el text i el model podem dir que, per al cas del model, aquest objectiu s'ha assolit amb un bon nivell de qualitat. Es calcula un ordre parcial bàsic amb el \emph{Behavioral Profile} i aquest ordre s'amplia amb la informació addicional dels \emph{Message Flows} utilitzant un algorisme personalitzat. Pel que fa a l'ordenació de frases, s'ha optat per un enfoc bàsic però el resultat final és efectiu.

L'algorisme d'optimització per trobar la correspondència entre tasques i frases ha estat un dels punts forts del projecte i on s'hi ha invertit més temps d'experimentació. Durant el desenvolupament s'han plantejat diversos tipus d'alternatives, de les quals LAP, CSP i ILP han estat les principals (secció \ref{sec:implementacio-correspondencia}). Utilitzant ILP, s'ha trobat una implementació que compleix les restriccions definides a l'apartat \ref{sec:enfoc-matching} i que a més ho fés amb un nivell d'eficiència molt superior a l'esperat. Per aquest fet, podem dir que el grau de satisfacció d'aquesta part és molt elevat. A més, s'han implementat dues altres alternatives també viables. En cas que creixés considerablement la mida del problema es pot utilitzar LAP a canvi d'una pèrdua d'optimalitat. D'altra banda, si cal introduir noves restriccions més complexes al problema, CP quedaria com una solució alternativa menys eficient però molt més expressiva.

Finalment, la detecció d'inconsistències s'ha realitzat de manera implícita. En comptes de retornar una llista d'errors a l'usuari, l'algorisme mostra l'assignació òptima calculada a l'usuari, oferint informació detallada de perquè ha assignat cadascuna de les tasques. Aquesta informació es pot utilitzar per assistir a l'usuari a l'hora de detectar problemes entre el model BPMN i la descripció textual (veure secció \ref{sec:experiments-inconsistencies}). 

Com a possibles millores a l'algorsime desenvolupat, possiblement la major contribució immediata a millorar la qualitat de l'algorisme és adreçar els problemes que sorgeixen durant la discussió a l'experiment de la secció \ref{sec:experiments-avaluacio}. Pot ser també molt interessant experimentar amb noves mètriques de similaritat i tipus de característiques. Finalment, una possible ampliació futura seria integrar un mètode més sofisticat per a l'ordenació temporal de frases en un text en llenguatge natural.

El tractament de processos de negoci utilitzant tècniques d'\emph{NLP} és un camp emergent. L'eina desenvolupada en aquest projecte, que per mesurar la similaritat entre una descripció textual i un model BPMN, pot resultar molt útil a l'hora de validar els resultats de diversos algorismes, com el de traducció automàtica de text a model BPMN. Per altra banda, aquesta eina es pot utilitzar per ajudar a detectar inconsistències de manera automàtica entre diferents tipus de documentació en una empresa real.  Aquest projecte, com la resta d'estudis en aquest camp, té per objectiu la creació d'eines que assisteixin a la presa de decisions i ajudin a optimitzar els processos de negoci. Aquest enfoc algorísmic al problema pot obrir la porta a millores considerables en el funcionament de les institucions i empreses.
